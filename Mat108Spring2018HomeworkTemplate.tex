\documentclass[12pt]{article}
\usepackage[margin=1in]{geometry}
\usepackage[all]{xy}
\usepackage{MnSymbol,wasysym}
\usepackage{enumitem}
\usepackage{enumerate}
\usepackage{amsmath,amsthm,amssymb,color,latexsym}
\usepackage{geometry}    
\usepackage{fancyvrb}
\usepackage{listings}
%----------------- shortcuts -----------------%
\newcommand{\R}{\mathbb{R}}
\newcommand{\N}{\mathbb{N}}
\newcommand{\Z}{\mathbb{Z}}
\newcommand{\Q}{\mathbb{Q}}
\newcommand{\ep}{\epsilon}
\newcommand{\epo}{\ep > 0}
\newcommand{\norm}[1]{\left\lVert#1\right\rVert}

\newenvironment{theorem}[2][Theorem]{\begin{trivlist}
\item[\hskip \labelsep {\bfseries #1}\hskip \labelsep {\bfseries #2.}]}{\end{trivlist}}
\newenvironment{lemma}[2][Lemma]{\begin{trivlist}
\item[\hskip \labelsep {\bfseries #1}\hskip \labelsep {\bfseries #2.}]}{\end{trivlist}}
\newenvironment{exercise}[2][Exercise]{\begin{trivlist}
\item[\hskip \labelsep {\bfseries #1}\hskip \labelsep {\bfseries #2.}]}{\end{trivlist}}
\newenvironment{reflection}[2][Reflection]{\begin{trivlist}
\item[\hskip \labelsep {\bfseries #1}\hskip \labelsep {\bfseries #2.}]}{\end{trivlist}}
\newenvironment{proposition}[2][Proposition]{\begin{trivlist}
\item[\hskip \labelsep {\bfseries #1}\hskip \labelsep {\bfseries #2.}]}{\end{trivlist}}
\newenvironment{corollary}[2][Corollary]{\begin{trivlist}
\item[\hskip \labelsep {\bfseries #1}\hskip \labelsep {\bfseries #2.}]}{\end{trivlist}}
\newenvironment{remark}[2][Remark]{\begin{trivlist}
\item[\hskip \labelsep {\bfseries #1}\hskip \labelsep {\bfseries #2.}]}{\end{trivlist}}
\newenvironment{example}[2][Example]{\begin{trivlist}
\item[\hskip \labelsep {\bfseries #1}\hskip \labelsep {\bfseries #2.}]}{\end{trivlist}}
    
\geometry{letterpaper}    
\usepackage{graphicx}

\newtheorem{problem}{Problem}

\newenvironment{solution}[1][\it{Solution}]{\textbf{#1. } }{$\smiley{}$}
\renewenvironment{proof}[1][\it{Proof}]{\textbf{#1. } }{$\smiley{}$}

    
\geometry{letterpaper}    
\usepackage{graphicx}

% DOCUMENT BEGINS HERE ---------------------------------:)
\begin{document}
%TITLE --------------------------------------
\noindent Fourier Analysis DRP Spring 2026\hfill Week \#2\\Rosie Y. (02/14)
%ACTUAL THING --------------------------------------
\hrulefill
\begin{example}{1}
    The $N^{th}$ Dirichlet kernel is the trigonometric polynomial defined for $x \in [-\pi,\pi]$ by 
    \[D_N(x)=\sum_{n=-N}^{N}e^{inx}\]
    A simple sum splitting yields
    \begin{align*}
        D_N(x)&=\sum_{n=-N}^{N}e^{inx}\\
        &=\sum_{n=0}^{N}e^{inx}+\sum_{n=-N}^{-1}e^{inx}
    \end{align*}
    Setting $\omega=e^{ix}$, we can write both of these as finite geometric series
    \[\sum_{n=0}^{N}e^{inx}=1+\omega+\dots+\omega^{N}=\left(\frac{1-\omega^{N+1}}{1-\omega}\right)\]
    \[\sum_{n=-N}^{-1}e^{inx}=\omega^{-1}+\dots+\omega^{-N}=\left(\frac{1}{w}\right)\left(\frac{1-\omega^{-N}}{1-\omega^{-1}}\right)=\left(\frac{1-\omega^{-N}}{\omega-1}\right)=\left(\frac{\omega^{-N}-1}{1-\omega}\right)\]
    Their sum is then
    \begin{align*}
        \sum_{n=-N}^{N}e^{inx}&=\left(\frac{1-\omega^{N+1}}{1-\omega}\right)+\left(\frac{\omega^{-N}-1}{1-\omega}\right)\\
        &=\left(\frac{1-\omega^{N+1}+\omega^{-N}-1}{1-\omega}\right)\\
        &=\left(\frac{\omega^{-N}-\omega^{N+1}}{1-\omega}\right)\\
        &=\left(\frac{\omega^{-1/2}}{\omega^{-1/2}}\right)\left(\frac{\omega^{-N}-\omega^{N+1}}{1-\omega}\right)\\
        &=\left(\frac{\omega^{-(N+1/2)}-\omega^{N+1/2}}{\omega^{-1/2}-\omega^{1/2}}\right)
    \end{align*}
    Using Euler's identities, we have
    \[\omega^{-(N+1/2)}=e^{-(N+1/2)xi}=\cos(-(N+1/2)x)+i\sin(-(N+1/2)xi)\]
    \[\omega^{N+1/2}=e^{(N+1/2)xi}=\cos((N+1/2)x)+i\sin((N+1/2)xi)\]
    \[\omega^{-(N+1/2)}-\omega^{N+1/2}=-2i\sin((N+1/2)xi)\]
    Similarly
    \[\omega^{-1/2}=e^{-(1/2)xi}=\cos(-(1/2)x)+i\sin(-(1/2)x)\]
    \[\omega^{1/2}=e^{(1/2)xi}=\cos((1/2)x)+i\sin((1/2)x)\]
    \[\omega^{-1/2}-\omega^{1/2}=-2i\sin((1/2)x)\]
    Hence
    \[\left(\frac{\omega^{-(N+1/2)}-\omega^{N+1/2}}{\omega^{-1/2}-\omega^{1/2}}\right)=\left(\frac{-2i\sin((N+1/2)xi)}{-2i\sin((1/2)x)}\right)=\frac{\sin((N+1/2)x)}{\sin(x/2)}\]
    This says that the Dirichlet kernel has the following closed form formula
    \[D_N(x)=\frac{\sin((N+1/2)x)}{\sin(x/2)}\]
    We shall now present a different argument that's a bit more motivating. Notice that given $f \in C_{per}(\R)$, its $N^{th}$ Fourier partial sum is given by
    \[\Pi_Nf=\frac{a_0}{2}+\langle \rangle\]
\end{example}
\end{document}
